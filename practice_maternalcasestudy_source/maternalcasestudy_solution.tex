%% LyX 2.3.6.2 created this file.  For more info, see http://www.lyx.org/.
%% Do not edit unless you really know what you are doing.
\documentclass[12pt]{article}
\usepackage[latin9]{inputenc}
\usepackage{amsmath}
\usepackage{amssymb}

\makeatletter
%%%%%%%%%%%%%%%%%%%%%%%%%%%%%% Textclass specific LaTeX commands.
\newenvironment{lyxlist}[1]
	{\begin{list}{}
		{\settowidth{\labelwidth}{#1}
		 \setlength{\leftmargin}{\labelwidth}
		 \addtolength{\leftmargin}{\labelsep}
		 \renewcommand{\makelabel}[1]{##1\hfil}}}
	{\end{list}}

%%%%%%%%%%%%%%%%%%%%%%%%%%%%%% User specified LaTeX commands.
\usepackage{amsfonts}
\title{Case Study: The Economics of Welfare Reform: Maternal Labor Supply and Child Academic Outcomes}

\makeatother

\begin{document}
\title{Case Study: Maternal Labor Supply and Child Academic Outcomes}
\maketitle

\section*{Background}

This case study focuses on how a single mother, Lisa, allocates her
time between work, child care, and leisure. The objective is to understand
how different welfare reform policies might affect both Lisa's labor
supply and her child's academic outcomes.

\section*{Scenario}

Meet Lisa, a single mother in her early 30s, who is contemplating
how to allocate her time between work, child care, and leisure as
she navigates various welfare reform policies. The time frame of the
decision is one week (168 hours).

\section*{Setting Up the Problem}

\subsection*{Given:}
\begin{itemize}
\item $c$ = Consumption goods. 
\item $L$ = Leisure time measured in hours. 
\item $m$ = Time spent on maternal care measured in hours. 
\item $l$ = Time spent on work measured in hours. 
\item $T$ = Total available time in a week, such that $m+l+L=T$ (typically
$T=168$ hours). 
\item $p$ = Price of consumption goods, assumed constant. 
\item $w$ = Wage rate per hour. 
\item $B$ = Welfare benefits (converted to weekly income). 
\item $\theta$ = Academic outcome of the child, given by $\theta=bm$ for
$b\in(0,1)$. 
\end{itemize}

\subsection*{Utility Function:}

Lisa's utility function $U(c,L,\theta)$ captures her preferences
for consumption, leisure, and her child's academic outcomes. The utility
function is:

\[
U(c,L,\theta)=c^{a}+L^{a}+\theta^{a}
\]

Here, $a\neq1$, is a preference parameter. 

\subsection*{Budget Constraint:}

Before Reform: Lisa consumes $c$ at prices $p$, given labor income
$W\times l$ before reform.

After Reform: Lisa consumes $c$ at prices $p$, given labor income
$W\times l+B$ after reform.

\subsection*{Time Constraint:}

Before and After Reform:

\[
m+l+L=T
\]


\section*{Questions}

\section{Without Welfare Reform}
\begin{enumerate}
\item \textbf{Budget and Time Constraints}: Write down Lisa's budget and
time constraints before welfare reform. Explain the meaning of the
budget. Explain how $\theta$ affects the utility of Lisa  $U(c,L,\theta)$.
Are the preference concave, monotone? (If yes, for what values of
$a$). 
\item \textbf{Optimization}: Use the Lagrangian method to solve for Lisa's
optimal choices of $c$, $L$, $m$, $\theta$ and $l$ before and
after reform. You can reduce the problem to $3$ variables but not
less than that. Given the utility function, you are asked to use,
all the optimal quantities will be positive. Call the optimal values
of this situation 1: $c^{1},L^{1},m^{1},\theta^{1},l^{1}$.
\end{enumerate}
Answer:

1.1. $pc+w(m+L)\leq wT$, the utility $U(c,L,\theta)=c^{a}+L^{a}+\theta^{a}$
it s a CES utility function it is concave for $a\leq1$ and and convex
for $a>1$, it is monotone as it increases its value with positive
quantities of consumption of $c,L,\theta$. 

1.2. I write down the problem in terms of $m$ replacing the equation
$\theta=bm$

$\mathcal{L}(c,L,m)=c^{a}+L^{a}+(bm)^{a}+(wT-pc-w(m+L))$

We proceed to obtain the derivatives of $\mathcal{L}$ with respect
to $c,L,m$:

$\frac{\partial\mathcal{L}}{\partial c}=ac^{-1+a}-p\lambda=0$

$\frac{\partial\mathcal{L}}{\partial L}=aL^{-1+a}-w\lambda=0$

$\frac{\partial\mathcal{L}}{\partial m}=ab(bm)^{-1+a}-w\lambda=0$

$\frac{\partial\mathcal{L}}{\partial\lambda}=wT-pc-w(m+L)=0$

Now, we proceed to find optimal values of $c,L,m$ in terms of the
parameters of the problem, t his is done by solving the system of
equatios implied by the FOC of the Lagrangian. 

In this particular case, echoing what we have done in class, we write
down two MRS conditions the first MRS$(L,c)$ and MRS$(m,c)$, then
use that to obtain $L$ in terms of $c$ and parameters, and $m$
in terms of $c$ and parameters. Then we replace back in the budget
constraint to obtain the optimal value of $c$ in terms of parameters
alone and then solve the whole problem.

The two MRS conditions are:

\[
c^{-1+a}L^{1-a}=\frac{p}{w}
\]

\[
c^{-1+a}m(bm)^{-a}=\frac{p}{w}
\]

From this I obtain:

\[
L=c\left(\frac{p}{w}\right)^{\frac{1}{1-a}}
\]

\[
m=c\left(\frac{b^{a}p}{w}\right)^{\frac{1}{1-a}}
\]

Then replace in the budget constraint:

\[
Tw-c\left(p+\left(\left(\frac{p}{w}\right)^{\frac{1}{1-a}}+\left(\frac{b^{a}p}{w}\right)^{\frac{1}{1-a}}\right)w\right)=0
\]

Obtain the optimal quantities, as you can see LHS is full of parameters
and no variable remains, also I have verified that this quantities
satisfy the budget contraint:

\[
c^{*1}=\frac{Tw}{p+\left(\left(\frac{p}{w}\right)^{\frac{1}{1-a}}+\left(\frac{b^{a}p}{w}\right)^{\frac{1}{1-a}}\right)w}
\]

\[
L^{*1}=\frac{Tw\left(\frac{p}{w}\right)^{\frac{1}{1-a}}}{p+\left(\left(\frac{p}{w}\right)^{\frac{1}{1-a}}+\left(\frac{b^{a}p}{w}\right)^{\frac{1}{1-a}}\right)w}
\]

\[
m^{*1}=\frac{Tw}{w+\left(\frac{b^{a}p}{w}\right)^{\frac{1}{-1+a}}\left(p+\left(\frac{p}{w}\right)^{\frac{1}{1-a}}w\right)}.
\]

\[
\theta^{*1}=b\frac{Tw}{w+\left(\frac{b^{a}p}{w}\right)^{\frac{1}{-1+a}}\left(p+\left(\frac{p}{w}\right)^{\frac{1}{1-a}}w\right)}
\]


\section{With the Welfare Reform}
\begin{enumerate}
\item Write down Lisa's budget and time constraints after the welfare reform
takes place. Explain the meaning of the budget. 
\item \textbf{Optimization}: Use the Lagrangian method to solve for Lisa's
optimal choices of $c$, $L$, $m$, and $l$ before and after reform.
You can reduce the problem to $3$ variables but not less than that.
Given the utility function, you are asked to use, all the optimal
quantities will be positive. Call the optimal values $c^{*2},L^{*2},m^{*2},\theta^{*2}$
and $l^{*2}$. 
\item \textbf{Policy Analysis}: Based on the results before the reform,
how generous should the welfare benefit $B$ be in order to double
the academic output ( $\theta^{*2}=2\theta^{*1})$? 
\end{enumerate}
Answer:

2.1. When Lisa has $T-L-m>0$ then the budget is $pc+w(m+L)\leq wT+B$,
when $T-L-m=0$ then the budget is $pc\leq B$. The explanation of
this budget can echo the analogous discussion in Chapter 5. 

2.2. $\mathcal{L}(c,L,m)=c^{a}+L^{a}+(bm)^{a}+(wT+B-pc-w(m+L))$

The solution for this case is completely analogous to the first part
so I will omit and only write down the solutions in your case put
all your steps. 

\[
c^{*2}=\frac{Tw+B}{p+\left(\left(\frac{p}{w}\right)^{\frac{1}{1-a}}+\left(\frac{b^{a}p}{w}\right)^{\frac{1}{1-a}}\right)w}
\]

\[
L^{*2}=\frac{(Tw+B)\left(\frac{p}{w}\right)^{\frac{1}{1-a}}}{p+\left(\left(\frac{p}{w}\right)^{\frac{1}{1-a}}+\left(\frac{b^{a}p}{w}\right)^{\frac{1}{1-a}}\right)w}
\]

\[
m^{*2}=\frac{Tw+B}{w+\left(\frac{b^{a}p}{w}\right)^{\frac{1}{-1+a}}\left(p+\left(\frac{p}{w}\right)^{\frac{1}{1-a}}w\right)}.
\]

We use the equations of $l^{*2}=T-L^{*2}-m^{*2}$, and essentially
assume that $T-L-m>0$, in general the CES produces interior solutions
so (for most parameters) this is fine. Then obtain $\theta^{*2}=bm^{*2}$. 

\[
l^{*2}=\frac{-B+\frac{p(B+Tw)}{p+\left(\left(\frac{p}{w}\right)^{\frac{1}{1-a}}+\left(\frac{b^{a}p}{w}\right)^{\frac{1}{1-a}}\right)w}}{w}
\]

\[
\theta^{*2}=\frac{b(B+Tw)}{w+\left(\frac{b^{a}p}{w}\right)^{\frac{1}{-1+a}}\left(p+\left(\frac{p}{w}\right)^{\frac{1}{1-a}}w\right)}
\]

2.3. The quantity of transfer needed to obtain a doubling of the academic
outcome is 

\[
\theta^{*1}=b\frac{Tw}{w+\left(\frac{b^{a}p}{w}\right)^{\frac{1}{-1+a}}\left(p+\left(\frac{p}{w}\right)^{\frac{1}{1-a}}w\right)}
\]

\[
\theta^{*2}=2b\frac{Tw}{w+\left(\frac{b^{a}p}{w}\right)^{\frac{1}{-1+a}}\left(p+\left(\frac{p}{w}\right)^{\frac{1}{1-a}}w\right)}
\]

\[
\frac{b(B+Tw)}{w+\left(\frac{b^{a}p}{w}\right)^{\frac{1}{-1+a}}\left(p+\left(\frac{p}{w}\right)^{\frac{1}{1-a}}w\right)}=2b\frac{Tw}{w+\left(\frac{b^{a}p}{w}\right)^{\frac{1}{-1+a}}\left(p+\left(\frac{p}{w}\right)^{\frac{1}{1-a}}w\right)}
\]

\[
B+Tw=2Tw
\]

\[
B=Tw.
\]

That is the wage time the total time $T$. 

\section{Policy Implications and Real-world connections.}
\begin{enumerate}
\item \textbf{Optimal Welfare Reform: }Assume the policy-maker aims to optimize
their own utility function $Z$ that reflects a weighted sum of the
log of household income (through labor) and the log of the academic
outcomes of children. For simplicity set $w=1$, $p=1$, $a=1/2,$
$b=1/2$ and $T=168$.
\end{enumerate}
Formally, this can be represented as: 

\[
Z(B)=(l^{*2}(B))^{\delta}(\theta^{*2}(B))^{(1-\delta)}
\]
 

where $\delta$ is a weight parameter $(0\leq\delta\leq1)$ that captures
the policy-maker's preference between the two objectives. 

Here, $l^{*2}(B)$ and $\theta^{*}(B)$ are the optimal labor supply
and academic outcome, respectively, that depend on $B$, as solved
in the earlier questions (situation 2). 

How should the policy-maker optimally choose $B$ to maximize $Z$?
Provide the mathematical formulation and solve it. Hint: This is equivalent
to maximizing a univariate concave function without constraints that
you learned in Calculus. Notice in this case there is no constraints
(we are assuming away government financing here). Optimize for $\delta=\frac{1}{3}$,
explain the value, and obtain the optimal transfer under this parameter
(it should be a positive quantity $B>0$), explain the economical
logic on this result. 

Answer:

Replacing all the parameters we obtain:

\[
l^{*2}(B)=\frac{2(168+B)}{5}-B
\]

\[
\theta^{*2}(B)=\frac{168+B}{10}
\]

Then we obtain for $\delta=\frac{1}{3}$ 

\[
Z(B)=\frac{1}{5}(336-3B)^{\frac{1}{3}}(84+\frac{B}{2})^{2/3}
\]

To maximize it we obtain its first derivative with respect to $B$,
and verify it is a concave function then:

\[
\frac{\partial Z}{\partial B}=\frac{56-3B}{5(672-6B)^{\frac{2}{3}}(168+B)^{\frac{1}{3}}}=0
\]

\[
B=\frac{56}{3}.
\]

\begin{lyxlist}{00.00.0000}
\item [{\textbf{2.}}] \textbf{Real-world Relevance: }Link the findings
of this model to the referenced paper on "Designing Cash Transfers
in the Presence of Children\textquoteright s Human Capital Formation."
from Mullins. Explain how this simplified model helps us understand
implications about the optimal welfare reform reported in this paper. 
\end{lyxlist}
This is the abstract of that paper: ``This paper finds that accounting
for the human capital development of children has a quantitatively
large effect on the true costs and benefits of providing cash assistance
to single mothers in the United States. A dynamic model of work, welfare
participation, and parental investment in children introduces a formal
apparatus for calculating costs and benefits when individuals respond
to incentives. The model provides a tractable outcome equation in
which a policy\textquoteright s effect on child skills can be understood
through its impact on two economic resources in the household --
time and money -- and the share of each resource as factors in the
production of skills. These key causal parameters are cleanly identified
by policy variation through the 1990s. The model also admits simple
and interpretable formulae for optimal nonlinear transfers in the
style of Mirrlees (1971), with novel features arising when child skill
formation is accounted for. Using a broadly conservative empirical
strategy, estimates imply that optimal transfers are about 20\% more
generous than the US benchmark, and shaped very differently. In contrast
to current policies, the optimal policy discourages labor supply at
the bottom of the income distribution due to the costly estimated
impacts of work on child development. The finding underscores the
importance of reconciling results in the literature on the developmental
effects of maternal employment. Finally, a counterfactual model exercise
suggests that changes to the welfare and tax environment after 1996
had negative average effects both on maternal welfare and child skill
outcomes, with a significant degree of redistribution across latent
dimensions.''

This part I will leave it for you to think but a good answer will
be concise yet able to convey your knowledge of the class, and the
understanding of the case. 
\end{document}
