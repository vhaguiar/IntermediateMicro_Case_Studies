%% LyX 2.3.6.2 created this file.  For more info, see http://www.lyx.org/.
%% Do not edit unless you really know what you are doing.
\documentclass[12pt]{article}
\usepackage[latin9]{inputenc}
\begin{document}
\title{Case Study: The Economics of Shopping: Time, Money, and Retirement}
\maketitle

\section*{Background}

This case study explores how an individual, Carol, allocates her time
between leisure, work, and shopping. The focus is on understanding
why Carol might consume more after retirement despite a potential
reduction in total income.

\section*{Scenario}

Meet Carol, a woman in her late 50s who is contemplating how to allocate
her time between work, leisure, and shopping as she approaches retirement.
The time frame of the decision is one day ($24$ hours).

\section*{Setting Up the Problem}

\textbf{Given:}
\begin{itemize}
\item $c$ = Consumption goods.
\item $L$ = Leisure time measured in hours. 
\item $s$ = Time spent on shopping measured in hours. 
\item $l$ = Time spent on work measured in hours. 
\item $T$ = Total available time in a day, such that $s+l+L=T$ (typically
$T=24$ hours). 
\item $p$ = Price of consumption goods, determined by $p=f(s)$, such that
(its derivative) $f'(s)<0$ capturing the idea that as more time is
spent shopping the better deals can the consumer obtain. We require
that even if the consumer spends all the time shopping the price is
positive ($f(T)>0$). 
\item $w$ = Wage rate per hour. 
\item $R$ = Retirement income (converted to daily income). 
\end{itemize}
\textbf{Budget Constraint:}

Still Working: Carol consumes $c$ at prices $f(s)$, given labor
income $w\times l$ before retirement. 

Carol retires: Carol consumes $c$ at prices $f(s)$, given retirement
income $R$ after retirement. The quantity of retirement she receives
does not depends on savings so its fully non labor income.

\textbf{Time Constraint:}

Still working:

\[
s+l+L=T.
\]

Carol retires:

\[
s+L=T.
\]


\section*{Questions}

\section{Situation 1: Carol is Still Working}
\begin{enumerate}
\item \textbf{Budget and Time Constraints}: Write down Carol's budget and
time constraints when she is still working. Assume $R=0$ in this
case. Hint: use $f(s)=(s)^{-b}$ for $b>0$. Explain in few sentences
the meaning of the shopping technology $f(s)$ in terms of the parameter
$b$ and interpret its economic meaning. 
\item \textbf{Optimization}: Use the Lagrangian method to solve for Carol's
optimal choices of $c$, $L$, $l$ and $s$ when she is working Show
all steps. Call the optimal value of this problem: $c^{1},L^{1},l^{1}$
and $s^{1}$. (Important: to solve the problem pick a utility function
$u(c,L)$, choose smart and on the basis of what you have learned.
Use any utility you want but: Use the fact that consumption goods
and leisure are neither perfect substitutes nor perfect complements.
Also make sure your preferences are convex, hence the utility is a
concave function. Finally, use a utility function that has only one
parameter and call this preference parameter $a$). 
\end{enumerate}
(Hint: you can simplify the problem to $3$ variables but there is
no way to do only $2$, when solving this Lagrangian notice that $\lambda>0$,
and think of ways to obtain the solution without having to compute
the value of $\lambda$). 

\section{Situation 2: Carol Retires}
\begin{enumerate}
\item \textbf{New Budget Constraint}: Write down Carol's new budget constraint
after retirement. Assume $R>0$. 
\item \textbf{Optimization}: Use the Lagrangian method again to solve for
Carol's optimal choices of $c$, $L$, and $s$ after retirement ($l^{2}=0$).
Show all steps. (You can reduce the problem here to $2$ variables
here) Call the optimal values of situation 2, $c^{2},L^{2},s^{2}$. 
\item \textbf{Consumption Puzzle}: Find the range of $R$ under which Carol
ends up consuming more ($\frac{c^{1}}{c^{2}}\leq1$) after retirement
than when she was working, despite $0<R\leq w\times l^{1}$, where
$l^{1}$ is the solution to situation 1 (Hint: set the value of other
other parameters distinct to $R$ to numbers: $w=1$, $a=\frac{1}{2}$,
and $b=\frac{1}{2}$, $T=24$). 
\end{enumerate}

\section{Policy Implications and Real-world Connections}
\begin{enumerate}
\item \textbf{Different Choices}: Discuss how Carol's choices in shopping
and leisure might differ before and after retirement, even though
her utility function remains the same. 
\item \textbf{Real-world Relevance}: Link the findings of this model to
the referenced paper on "Life-Cycle Prices and Expenditure." from
Aguiar and Hurst. Explain how this simplified model helps us understand
real-world behaviors observed in the paper. This is the abstract of
that paper: ``Previous authors have documented a dramatic decline
in food expenditures at the time of retirement. We show that this
is matched by an equally dramatic rise in time spent shopping for
and preparing meals. Using a novel data set that collects detailed
food diaries for a large cross section of U.S. households, we show
that neither the quality nor the quantity of food intake deteriorates
with retirement status. We also show that unemployed households experience
a decline in food expenditure and food consumption commensurate with
the impact of job displacement on permanent income.''
\end{enumerate}
Your answer should be a well done one paragraph.

\pagebreak{}

\part*{Answer to Case Study: The Economics of Shopping: Time, Money, and
Retirement}

\section*{1. Answer: Situation 1 Carol is Still Working}

1.1 The budget is $f(s)c\leq wl$ and $s+l+L=T$, now we require $l=T-s-L$,
and $f(s)c\leq w(T-s-L)$, and rearranging terms

\[
f(s)c+w(s+L)\leq wT.
\]

1.2. I pick the utility function $u(c,L)=alog(c)+(1-a)log(L)$, for
$a\in(0,1)$, and $f(s)=(s+1)^{-b}$ for $b\in(0,1)$. 

The budget is 

\[
(s+1)^{-b}c+w(s+L)\leq wT.
\]

The Lagrangian is:

\[
\mathcal{L}(c,L,s)=alog(c)+(1-a)log(L)+\lambda(wT-(s+1)^{-b}c-w(s+L))
\]

$\frac{\partial\mathcal{L}}{\partial c}=\frac{a}{c}-\lambda[(s)^{-b}]=0$

$\frac{\partial\mathcal{L}}{\partial L}=\frac{(1-a)}{L}-\lambda w=0$

$\frac{\partial\mathcal{L}}{\partial s}=\lambda(bc(s)^{-1-b}-w)=0$

$\frac{\partial\mathcal{L}}{\partial\lambda}=wT-(s+1)^{-b}c-w(s+L)=0$

Using $\frac{\partial\mathcal{L}}{\partial c}$ and $\frac{\partial\mathcal{L}}{\partial L}$
we can form the MRS condition:

\[
\frac{a}{1-a}\frac{L}{c}=\frac{s^{-b}}{w}
\]

From this we obtain the first equation:

\[
L=\frac{(1-a)}{a}\frac{cs^{-b}}{w}
\]

Now from $\frac{\partial\mathcal{L}}{\partial s}$ we know that since
$\lambda>0$ this means that:

\[
bc(s)^{-1-b}-w=0
\]

This means:

\[
c=\frac{s^{1+b}w}{b}.
\]

Now we obtain a second equation from $L$:

\[
L=\frac{(1-a)}{a}\frac{s}{b}.
\]

We replace the expressions above of $c$ and $L$ written in terms
of $s$ and parameters $a,b$ in the budget, and we obtain the optimal
shopping:

\[
s^{1}=\frac{ab}{(1+ab)}T
\]

\[
L^{1}=\frac{(1-a)}{(1+ab)}T
\]

\[
c^{1}=\frac{\left(\frac{ab}{(1+ab)}T\right)^{1+b}w}{b}
\]

\[
l^{1}=\frac{a}{(1+ab)}T.
\]


\section*{2. Answer: Situation 2 Carol is Retired }

2.1. The new budget is $f(s)c\leq R$, $s+L=T$, in this case we will
let $L=T-s$, and plug in the utility to form the Lagrangian. 

2.2. The new lagrangian is the following:

\[
\mathcal{L}_{2}(c,s)=aLog(c)+(1-a)Log(T-s)+\lambda(R-s^{-b}c)
\]

The partial derivatives of lagrangians are:

$\frac{\partial\mathcal{L}_{2}}{\partial c}=\frac{a}{c}-\lambda s^{-b}=0$

$\frac{\partial\mathcal{L}_{2}}{\partial s}=-\frac{(1-a)}{T-s}+\lambda bcs^{-1b}=0$

$\frac{\partial\mathcal{L}_{2}}{\partial\lambda}=R-cs^{-b}=0$

Forming the MRS condition from $\frac{\partial\mathcal{L}_{2}}{\partial c}$,
$\frac{\partial\mathcal{L}_{2}}{\partial s}$ we obtain, the value
of $s$:

$\frac{ab(T-s)}{(1-a)s}=1$ meaning $s^{2}=\frac{ab}{(1-a+ab)}T$. 

Replacing this into the budget:

$c^{2}=R\left(\frac{ab}{(1-a+ab)}T\right)^{b}$. 

$L^{2}=\frac{T}{1+ab}.$

2.3. Consumption puzzle:

Set $w=1,a=\frac{1}{2},b=\frac{1}{2}$, $T=24$ compute 

\[
\frac{c^{1}}{c^{2}}=\frac{48\sqrt{\frac{3}{5}}}{5R}
\]

This means that $\frac{c^{1}}{c^{2}}\leq1\iff\frac{48\sqrt{\frac{3}{5}}}{5R}\leq1\iff R\geq\frac{48\sqrt{\frac{3}{5}}}{5}=7.43613$
and we verify that this number is below $wl^{1}=\frac{48}{5}=9.6$,
so there is the range of $R$

\[
\frac{48\sqrt{\frac{3}{5}}}{5}\leq R\leq\frac{48}{5}.
\]

For this range of $R$ even if the nonlabor retirement income of Carol
is lower than its labor income in situation 1, Carol consumes more
in situation 2. 

\section*{3. Answer: Policy Implications and Real-world Connections. }

\textbf{3.1 Different Choices}: Discuss how Carol's choices in shopping
and leisure might differ before and after retirement, even though
her utility function remains the same. 

In the analysis above, we observe that Carol when facing a different
budget and has no obligation to work but instead receives the nonlabor
income $R$, even when it is smaller than the salary $wl^{1}$, may
end up with Carol consuming more of $c$, of course, this happens
because she now can spend more time shopping, this leads to smaller
prices of consumption good, allowing her to buy more of $c$. 

\textbf{3.2. Real-world Relevance}: Link the findings of this model
to the referenced paper on "Life-Cycle Prices and Expenditure."
from Aguiar and Hurst. Explain how this simplified model helps us
understand real-world behaviors observed in the paper. This is the
abstract of that paper: ``Previous authors have documented a dramatic
decline in food expenditures at the time of retirement. We show that
this is matched by an equally dramatic rise in time spent shopping
for and preparing meals. Using a novel data set that collects detailed
food diaries for a large cross section of U.S. households, we show
that neither the quality nor the quantity of food intake deteriorates
with retirement status. We also show that unemployed households experience
a decline in food expenditure and food consumption commensurate with
the impact of job displacement on permanent income.''

Our model shows that indeed, it is possible for Carol to consumer
the same amount or more than before, while spending less, because
she spends more time shopping when retired thus finding good deals
and paying less. We did not model home production of goods, but we
modeled the shopping technology that decreased as the shopping time
increased. 
\end{document}
