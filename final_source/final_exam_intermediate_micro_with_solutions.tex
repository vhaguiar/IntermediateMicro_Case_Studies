%% LyX 2.3.6.2 created this file.  For more info, see http://www.lyx.org/.
%% Do not edit unless you really know what you are doing.
\documentclass{article}
\usepackage[utf8]{inputenc}

\makeatletter
%%%%%%%%%%%%%%%%%%%%%%%%%%%%%% Textclass specific LaTeX commands.
\newenvironment{lyxlist}[1]
	{\begin{list}{}
		{\settowidth{\labelwidth}{#1}
		 \setlength{\leftmargin}{\labelwidth}
		 \addtolength{\leftmargin}{\labelsep}
		 \renewcommand{\makelabel}[1]{##1\hfil}}}
	{\end{list}}

\@ifundefined{date}{}{\date{}}
%%%%%%%%%%%%%%%%%%%%%%%%%%%%%% User specified LaTeX commands.
\title{Gravity Model of Trade: Case Study of Arabasta and Baltigo}
\author{Victor H. Aguiar}

\makeatother

\begin{document}
\title{Final Exam: Case Study on Determinants of Bilateral Trade}
\maketitle

\section*{Introduction}

The Law of Gravity of Trade is a fundamental principle in international
economics, positing that trade between two countries is directly proportional
to their economic size, often measured by GDP, and inversely proportional
to the geographic distance between them. This model, akin to Newton's
law of gravity, suggests that larger economies have a greater gravitational
pull in trade relationships, while distance acts as a trade barrier,
diminishing trade interactions. This case study aims to apply this
principle in a simplified scenario involving two fictional countries,
Arabasta and Baltigo, to illustrate the microfoundations of the law
of gravity of trade.

\section*{Part 1: Consumption Problem for Arabasta}

We want to model the consumption of domestic and foreign products
for the country of Arabasta. 

\textbf{Model Setup:} 
\begin{itemize}
\item There are 2 countries, Arabasta (a) and Baltigo (b). Goods are differentiated
by their country of origin. 
\item $c_{b,a}$ denotes the consumption in Arabasta of the good produced
in Baltigo, and $c_{a,a}$ represents local consumption in Arabasta
of the good produced in Arabasta. 
\item Arabasta total revenue for selling his good both to itself and to
Baltigo is $y_{a}$. 
\item The prices faced by Arabasta for the good originating in Baltigo is
$p_{b,a}$, and the price for the good originating in Arabasta is
$p_{a,a}$. 
\end{itemize}

\subsubsection*{\emph{Task for the Student} }
\begin{lyxlist}{00.00.0000}
\item [{1.1.}] Write down a utility function for the representative consumer
of Arabasta, $u_{a}(c_{a,a},c_{b,a})$, and make sure this utility
has one parameter $\sigma$. Also write down the budget constraint
faced by Arabasta. 
\item [{1.2.}] Write down the Lagrangian associated with the constrainted
maximization of the utility and budget you wrote down before and obtain:
the consumption levels at Arabasta, $c_{b,a}^{*}$ and $c_{a,a}^{*}$.
(Notice that by carefully relabeling this optimal consumptions to
$c_{a,b}^{*}$ and $c_{b,b}^{*}$ you have obtained as well the Baltigo's
optimal consumption with the exact same utility parameter $\sigma$.).
\end{lyxlist}

\section*{Part 2: Market Clearing Conditions}

\textbf{Equilibrium Conditions:} 
\begin{itemize}
\item For market equilibrium: the total self-exports ($p_{a,a}$ times consumption
in Arabasta of Arabasta good), and the export to Baltigo ($p_{a,b}$
times consumption in Baltigo of Arabasta good) must equal Arabasta's
revenue, $y_{a}$.
\item The analogous case holds for Baltigo revenue $y_{b}$. 
\end{itemize}
\textbf{Price Formulation:} 
\begin{itemize}
\item Prices $p_{a,j}$ depend on base prices at origin, $p_{a}$, and trade
costs, $t_{a,j}$ for $j\in\{a,b\}$: $p_{a,b}=p_{a}t_{a,b}$ and
$p_{a,a}=p_{a}t_{a,a}$. Note that $t_{a,b}$ reflects the trade cost
due to distance and other barriers. We restrict that $t_{a,a}=1$,
so we can simplify it in the analysis, we also assume $t_{a,b}>1$
and $t_{a,b}=t_{b,a}$. In addition, we will assume that the only
trade barrier is distance between $a,b$, $d_{a,b}$.
\end{itemize}
\textbf{Trade Costs and Consumption:} 
\begin{itemize}
\item The cost of importing goods from Baltigo to Arabasta increases with
distance, affecting the prices and, consequently, the consumption
patterns. 
\end{itemize}

\subsubsection*{\emph{Task for the Student} }
\begin{lyxlist}{00.00.0000}
\item [{2.1.}] Write down the equilibrium condition that equates Arabasta's
revenue $y_{a}$ to its exports as described above. Taking as given
$t_{a,a},t_{a,b}$, and using the optimal demands of Part 1 $c_{a,a}^{*}$
and $c_{a,b}^{*}$ find the equilibrium value of the good in Arabasta,
$p_{a}^{*}$, using the market equilibrium conditions. Make sure that
this price is a function of $y_{a}$ and $t_{a,b}$ as well as the
parameter of preferences $\sigma$. (Hint: You have to normalize $p_{b}^{*}$,
e.g., $p_{b}^{*}=1$. Here we need an explicit solution and, if you
must, choose a number for your preference parameter. Note that there
is only one solution to this problem, if you are obtaining more than
one solution one of the possible solutions may not satisfy that $p_{a}>0$.
). 
\item [{2.2.}] Write down a production function, with a single parameter
$\alpha$, with non increasing returns for transporting goods and
services a certain distance $d_{a,b}=f(k)$, using $k$ units of capital
(ships). The rental rate (price of $k$) is $r=1$. For simplicity,
we assume here that neither Arabasta or Baltigo own this transport
company so the profits of this firm do not enter their budgets. Find
a cost function $c(d_{a,b})$ associated with the transportation production
function. Write down an equation for bilateral trade barriers $t_{a,b}=c(d_{a,b})+1$.
Make sure to verify that the larger the distance the larger the trade
barrier, make sure that when the distance is zero then $t_{a,b}=f(0)=1$.
Verify, using derivation, that as distance grows so does the trade
barrier. 
\end{lyxlist}

\section*{Part 3: Comparative Statics Analysis in Equilibrium}

 
\begin{itemize}
\item As the distance between Arabasta and Baltigo increases, the trade
cost $t_{ab}$ rises, leading to a decrease imports.
\item Bilateral trade is more substantial when both countries are economically
larger, even if the distance remains constant. 
\end{itemize}

\subsubsection*{Task for the Student}
\begin{lyxlist}{00.00.0000}
\item [{3.1}] Once you have obtained the prices $p_{a}^{*}$ and $p_{b}^{*}=1$
from the good of Arabasta and Baltigo: Obtain the consumptions in
Arabasta of Baligo's good in equilibrium, $c_{b,a}^{**}$ (Hint: you
can do this by pluggin in $p_{a}^{*},p_{b}^{*}$=1 from Part 2 into
the expression you obtained in Part 1 for $c_{a,b}^{*}$). Now, do
the same above, to obtain the nominal imports of Baltigo's good in
Arabasta $x_{b,a}^{**}=p_{b,a}^{*}c_{b,a}^{**}$ (make sure to replace
$p_{ab}^{*}$ by the appropiate terms given the previous analysis
using $p_{a}^{*}$). Notice that equilibrium quantities and equilibrium
imports should not longer be a function fo prices but only of $y_{a},y_{b},t_{a,b}(d_{a,b})$
. 
\item [{3.2.}] Demostrate the law of gravity of trade in your model. The
law of gravity of trade works at the equilibrium nominal imports $x_{b,a}^{**}$.
First, show that if we duplicate both $y_{a}$ and $y_{b}$ (you may
assume specific values for GDP, ($y_{a},$$y_{a}$) of Arabasta and
Baltigo, and distance) then $x_{b,a}^{**}$ strictly increases as
well. Second, by pluggin in your function for bilateral trade barrier
in terms of distance, show that if distance is larger, keeping the
sizes of Arabasta and Baltigo fixed, then $x_{b,a}^{**}$ strictly
decreases (you can also plug in numerical values here). 
\end{lyxlist}

\section*{Part 4: Microeconomic Foundation of the Law of Gravity of Trade}
\begin{description}
\item [{4.1.}] Provide a one paragraph explanation of why the Law of Gravity
of Trade explains bilateral trade flows between two countries using
the concepts we learned in our class and the mode you have developed.
Hint: Chapter 1 to 4 and Chapter 8 are the relevant parts here. 
\end{description}

\section*{Solutions}

\subsection*{Solution Part 1:}
\begin{lyxlist}{00.00.0000}
\item [{1.1.}] The utility is $u_{a}(c_{a,a},c_{b,a})=c_{a,a}^{\sigma}+c_{b,a}^{\sigma}$
for $\sigma$ the only parameter. The budget is $p_{aa}c_{aa}+p_{ba}c_{ba}=y_{a}$,
total imports (self imports + imports from Baltigo) equal to total
revenue. 
\end{lyxlist}
\begin{description}
\item [{1.2.}] The first step is to pose the Lagrangian
\end{description}
$L(c_{a,a},c_{b,a},\lambda)=c_{a,a}^{\sigma}+c_{b,a}^{\sigma}+\lambda(y_{a}-p_{aa}c_{aa}+p_{ba}c_{ba})$

(I) $\frac{\partial L}{\partial c_{a,a}}=\sigma c_{a,a}^{\sigma-1}-\lambda p_{a,a}=0$

(II) $\frac{\partial L}{\partial c_{b,a}}=\sigma c_{b,a}^{\sigma-1}-\lambda p_{b,a}=0$

(III) $\frac{\partial L}{\partial\lambda}=y_{a}-p_{aa}c_{aa}+p_{ba}c_{ba}=0$

From (I) and (II) we obtain the MRS equal to the ratio of prices conditions:

\[
\frac{c_{a,a}^{\sigma-1}}{c_{b,a}^{\sigma-1}}=\frac{p_{a,a}}{p_{b,a}}
\]

From this condition we isolate $x_{1}$:

\[
c_{a,a}=c_{b,a}\left(\frac{p_{a,a}}{p_{b,a}}\right)^{\frac{1}{\sigma-1}}
\]

Then we plug-in this into (III)/budget:

\[
p_{a,a}\left(c_{b,a}\left(\frac{p_{a,a}}{p_{b,a}}\right)^{\frac{1}{\sigma-1}}\right)+p_{b,a}c_{b,a}=y_{a}.
\]

Now solve for $c_{b,a}$ in terms of $p_{a,a},p_{b,a},y_{a}$:

\[
c_{b,a}^{*}=\frac{p_{b,a}^{\frac{1}{\sigma-1}}y_{a}}{p_{a,a}^{\frac{\sigma}{\sigma-1}}+p_{b,a}^{\frac{\sigma}{\sigma-1}}}
\]

\[
c_{a,a}^{*}=\frac{p_{a,a}^{\frac{1}{\sigma-1}}y_{a}}{p_{a,a}^{\frac{\sigma}{\sigma-1}}+p_{b,a}^{\frac{\sigma}{\sigma-1}}}.
\]

Baltigo Solution:

\[
c_{a,b}^{*}=\frac{p_{a,b}^{\frac{1}{\sigma-1}}y_{b}}{p_{b,b}^{\frac{\sigma}{\sigma-1}}+p_{a,b}^{\frac{\sigma}{\sigma-1}}}
\]

\[
c_{b,b}^{*}=\frac{p_{b,b}^{\frac{1}{\sigma-1}}y_{b}}{p_{b,b}^{\frac{\sigma}{\sigma-1}}+p_{a,b}^{\frac{\sigma}{\sigma-1}}}.
\]


\subsection*{Solution Part 2:}

2.1 Equilibrium equation says: 

\[
y_{a}=p_{a,a}c_{a,a}+p_{a,b}c_{a,b}.
\]

\[
y_{a}=p_{a}c_{a,a}+p_{a}t_{a,b}c_{a,b}
\]

Now, we plug in $c_{a,a}^{*},c_{a,b}^{*}$

\[
y_{a}=p_{a}\left(\frac{p_{a,a}^{\frac{1}{\sigma-1}}y_{a}}{p_{a,a}^{\frac{\sigma}{\sigma-1}}+p_{b,a}^{\frac{\sigma}{\sigma-1}}}\right)+p_{a}t_{a,b}\left(\frac{p_{a,b}^{\frac{1}{\sigma-1}}y_{b}}{p_{b,b}^{\frac{\sigma}{\sigma-1}}+p_{a,b}^{\frac{\sigma}{\sigma-1}}}\right)
\]

Replace the price expressions

\[
y_{a}=p_{a}^{\frac{\sigma}{\sigma-1}}\left(\frac{y_{a}}{p_{a}^{\frac{\sigma}{\sigma-1}}+t_{a,b}^{\frac{\sigma}{\sigma-1}}}+\frac{t_{a,b}^{\frac{\sigma}{\sigma-1}}y_{b}}{1+p_{a}^{\frac{\sigma}{\sigma-1}}t_{a,b}^{\frac{\sigma}{\sigma-1}}}\right)
\]

Letting $\sigma=\frac{1}{2}$ then, we can solve explicitly the implicit
solution above. In this case there are two numerical solutions, but
only one economic solution:

\[
p_{a}^{*}=\frac{y_{b}-y_{a}+\sqrt{y_{a}^{2}-2y_{a}y_{b}+4t_{a,b}^{2}y_{a}y_{b}+y_{b}^{2}}}{2t_{a,b}y_{a}}
\]

You can verify this is the unique nonnegative solution. 
\begin{lyxlist}{00.00.0000}
\item [{2.2.}] The cost function with $d_{a,b}=\frac{1}{\alpha}k$, this
is constant returns, for $\alpha>0$, then by inversion of the production
function $c(d_{a,b})=\alpha d_{a,b}$ with $r=1$. Then $t_{a,b}=\alpha d_{a,b}+1$,
$\frac{\partial t_{a,b}}{\partial d_{a,b}}=\alpha>0$.
\end{lyxlist}

\subsection*{Part 3 Solution}
\begin{lyxlist}{00.00.0000}
\item [{3.1}] We obtain $c_{b,a}^{**}$ by replacing 
\item [{
\[
c_{b,a}^{*}=\frac{p_{b,a}^{\frac{1}{\sigma-1}}y_{a}}{p_{a,a}^{\frac{\sigma}{\sigma-1}}+p_{b,a}^{\frac{\sigma}{\sigma-1}}},
\]
}]~
\item [{with}] the appropiate terms:
\item [{
\[
c_{b,a}^{*}=\frac{t_{a,b}^{\frac{1}{\sigma-1}}y_{a}}{p_{a}^{\frac{\sigma}{\sigma-1}}t_{a,a}^{\frac{\sigma}{\sigma-1}}+t_{a,b}^{\frac{\sigma}{\sigma-1}}},
\]
}]~
\item [{now}] replacing $p_{a}^{*}$ for $\sigma=\frac{1}{2}$, we obtain:
\item [{
\[
c_{b,a}^{**}=\frac{2y_{a}^{2}}{(-1+2t_{a,b}^{2})y_{a}+y_{b}+\sqrt{y_{a}^{2}-2y_{a}y_{b}+4t_{a,b}y_{a}y_{b}+y_{b}^{2}}},
\]
}]~
\item [{by}] multiplying by $p_{a}^{*}t_{a,b}$ we obtain $x_{a,b}^{**}$
\item [{
\[
x_{a,b}^{**}=\frac{y_{a}+y_{b}-\sqrt{y_{a}^{2}-2y_{a}y_{b}+4t_{a,b}^{2}y_{a}y_{b}+y_{b}^{2}}}{2-2t_{a,b}^{2}}
\]
}]~
\item [{3.2.}] $y_{a}=1,y_{b}=1,d_{a,b}=1$, with $\alpha=1$
\item [{
\[
x_{ab}^{**}=\frac{1}{3}
\]
}]~
\item [{
\[
y_{a}^{*}=2,y_{b}^{*}=2,d_{a,b}=1
\]
}]~
\item [{
\[
x_{ab}^{**}=\frac{2}{3}
\]
}]~
\item [{
\[
y_{a}=1,y_{b}=1,d_{a,b}=3
\]
}]~
\item [{
\[
x_{ab}^{**}=\frac{1}{5}.
\]
}]~
\end{lyxlist}

\subsection*{Part 4 Solution}

The effect of the size of countries is driven by the income effect
since both goods are normal as the income increases in both countries
demand will go up. The effect of distance is driven by the substitution
effect as distance increases prices of the imported good hence this
will drive demand down, but these findings are only possible because
imports is equal to export revenues, which is reminicisent of the
Edgeworth effect, due to that increases in prices will have a secondary
income effect, so the total effect is ambiguous but we have shown
using numerical examples that the gravity of bilateral trade works
out in our model as it does in real life. 
\end{document}
