%% LyX 2.3.6.2 created this file.  For more info, see http://www.lyx.org/.
%% Do not edit unless you really know what you are doing.
\documentclass{article}
\usepackage[latin9]{inputenc}
\usepackage{amstext}

\makeatletter
%%%%%%%%%%%%%%%%%%%%%%%%%%%%%% User specified LaTeX commands.
\usepackage{graphicx}% Required for inserting images

\title{The Economics of Superstars}
\author{Victor Aguiar}
\date{November 2023}

\makeatother

\begin{document}
\title{Case Study: The Economics of Superstars }
\maketitle

\section*{Introduction}

This case study, inspired by Krueger's 2015 research on the economics
of the music industry, delves into the intriguing relationship between
the sale of physical media (CDs) and live performance revenues (concerts).
The focal point is Taylor S., a superstar artist, and how market forces
and production constraints influence her pricing strategies for CDs
and concert tickets.

\section*{Part 1: Consumer Behavior Analysis}

\textbf{Context:} 
\begin{itemize}
\item A consumer, a fan of Taylor S., is deciding how to spend their budget
between music streaming services ($s$) and concert tickets ($c$)
produced by Taylor S. For simplicity we will assume that the consumer
can buy any fractional quantity of streaming services or concerts. 
\item The consumer faces a linear budget constraint with prices $p_{s}$
for CDs and $p_{c}$ for concerts, and a fixed income of $M=1$. 
\end{itemize}
\textbf{Task:} 
\begin{itemize}
\item Question 1.1.: Write down the utility function of the consumer (Taylor
S's fan) and their budget. 
\item Question 1.2. Obtain the optimal demand of CDs and concerts in terms
of prices $p_{s}$,$p_{c}$ (call these optimal quantities $s^{*}(p_{s},p_{c})$
and $c^{*}(p_{s},p_{c})$). 
\end{itemize}

\section*{Part 2: Taylor S.'s Production Function}

\textbf{Context:} 
\begin{itemize}
\item Taylor S.'s production functions for concerts ($y_{c}$) and songs
($y_{s})$ are given by $y_{c}=f^{c}(l)$ and $y_{s}=f^{s}(l)$ with
$l$ representing hours of work. 
\item The cost of working $l$ hours is $w$. 
\item Both production functions exhibit decreasing returns to scale. 
\end{itemize}
\textbf{Task:} 
\begin{itemize}
\item 2.1. Formulate Taylor S.'s production functions with decreasing returns
to scale for the production of concerts and songs. 
\item 2.2. Obtain the associated cost function for the production of songs
$cost^{s}(w,y_{s})$, and separately the cost function for the production
of concerts $cost^{c}(w,y_{c})$. Notice that for simplicity we are
assuming that this two production process are independent of each
other and that they face the same labor costs. 
\end{itemize}

\section*{Part 3: Profit Maximization Strategy}

\textbf{Context:} 
\begin{itemize}
\item Taylor S. operates as a monopolist in the concert market (sets $p_{c}$),
you can set the price of songs to $p_{s}=1$. A monopolist can set
up the price of concerts.
\end{itemize}
\textbf{Task:} 
\begin{itemize}
\item 3.1 Write down explicitly Taylor S's revenue: $Revenue(p_{s},p_{c},\tau)=p_{s}\cdot\tau s^{*}(p_{s},p_{c})+p_{c}\cdot c^{*}(p_{s},p_{c})$,
using the results from Part 1 ($c^{*}(p_{s},p_{c}),$$s^{*}(p_{s},p_{c})$).
(Notice that in this market $y_{c}=c$ and $y_{s}=\tau s$ where $\tau\in(0,1)$
is a fraction of the streaming services songs that belong to Taylor
S., and is set by a third party (Spotify)). Set $p_{s}=1$, and make
sure your choices of utility specification and parameters are such
that marginal revenue decreases with the price of concerts going up,
if it doesn't reconsider the choices of the utility specification. 
\item 3.1. Solve Taylor S.'s profit maximization problem using 3.1 and Part
2 ($cost^{c}(w,y_{c})$, $cost^{s}(w,y_{s})$). Set the price of songs,
$p_{s}=1$, set $w=1$. Solve for the equilibrium value of $p_{c}$
(it can be in implicit form):
\[
\pi(\tau,p_{s},w)=\max_{p_{c}}\left(Revenue(p_{s},p_{c},\tau)-cost^{s}(w,\tau s)-cost^{c}(w,c^{*}(p_{s,}p_{c}))\right)
\]
\end{itemize}

\section*{Part 4: Price Dynamics and Production }

\textbf{Context:} 
\begin{itemize}
\item An observed trend in the market is the rising prices of concert tickets
as $\tau$ decreases. A decrease of $\tau$ captures the idea that
the revenue Taylor S earns from songs decreases due to the distribution
systems such as Spotify. 
\end{itemize}
\textbf{Task:} 
\begin{itemize}
\item 4.1. In your model find parameters such that a decrease of $\tau$
(e.g., $\tau=1$ as a baseline and then move to $\tau=1/2$) leads
to an increase in concert ticket prices. Here you are encouraged to
plug-in numbers and you will need some help solving a polynomial equations
(scientific calculator, Wolfram Mathematica, among others can do this). 
\item 4.2. Discuss, in one paragraph, the economic rationale behind this
inverse relationship of $p_{c}$ and $\tau$, focusing on the production
constraints and decreasing returns in both concert and song production
but also discuss the role of market power yielded by Taylor S. in
the concert market contrasting it with her lack of power on the song
distribution side captured by $\tau$. 
\end{itemize}

\part*{Solution}

\section*{Part 1. Solution}

1.1. Let the utility be $u(s,c)=s^{\alpha}+c^{\alpha}$ subject $p_{s}s+p_{c}c=1$.

1.2. Choosing $\alpha=\frac{1}{2}$: 

We pose the lagrangian of the problem:

\[
L=s^{\frac{1}{2}}+c^{\frac{1}{2}}+\lambda(1-p_{s}s-p_{c}c)
\]

This lead to $s^{*}(p_{s},p_{c})=\frac{p_{c}}{p_{s}(p_{c}+p_{s})}$,
$c^{*}(p_{s},p_{c})=\frac{p_{s}}{p_{c}(p_{c}+p_{s})}$. 

\section*{Part 2. Solution.}

2.1. The following production functions are decreasing returns:

$y_{c}=l^{a_{c}}$, $y_{s}=l^{a_{s}}$, $a_{s},a_{c}\in(0,1)$. 

2.2. Cost can be obtained by inverting the prodution function so that 

$cost^{c}(w,y_{c})=wf^{c,-1}(y_{c})$ and $cost^{s}(w,y_{s})=wf^{s,-1}(y_{s})$,
for this particular specification:

$cost^{c}(w,y_{c})=w(y_{c})^{1/a_{c}}$, $cost^{s}(w,y_{s})=w(y_{s})^{1/a_{s}}$. 


\section*{Part 3. Solution.}

3.1. The revenue function is:

\[
Revenue(p_{s},p_{c},\tau)=p_{s}\cdot\tau s^{*}(p_{s},p_{c})+p_{c}\cdot c^{*}(p_{s},p_{c}),
\]

\[
Revenue(p_{s},p_{c},\tau)=p_{s}\tau\frac{p_{c}}{p_{s}(p_{c}+p_{s})}+p_{c}\frac{p_{s}}{p_{c}(p_{c}+p_{s})}
\]

\[
Revenue(p_{s},p_{c},\tau)=\tau\frac{p_{c}}{p_{c}+p_{s}}+\frac{p_{s}}{p_{c}+p_{s}}=\frac{\tau p_{c}+p_{s}}{p_{c}+p_{s}}
\]

Let $p_{s}=1$, then:

\[
Revenue(p_{c},\tau)=\frac{\tau p_{c}+1}{p_{c}+1}
\]

The marginal revenue is:

\[
\frac{\partial Revenue(p_{c},\tau)}{\partial p_{c}}=\frac{-1+\tau}{(1+p_{c})^{2}}<0.
\]

3.2. Profit:

\[
\pi(\tau,p_{s},w)=\max_{p_{c}}\left(Revenue(p_{s},p_{c},\tau)-cost^{s}(w,\tau s)-cost^{c}(w,c^{*}(p_{s,}p_{c}))\right)
\]

\[
\pi(\tau,p_{s},w)=\max_{p_{c}}\left(\frac{\tau p_{c}+1}{p_{c}+1}-\left(\frac{\tau p_{c}}{(p_{c}+1)}\right)^{1/a_{s}}-\left(\frac{1}{p_{c}(p_{c}+1)}\right)^{1/a_{c}}\right)
\]

Maximization happens when the first order conditions of profits are
equal to zero or equivalently when marginal revenue is equal to marginal
costs:

\[
\frac{-1+\tau}{(1+p_{c})^{2}}=\frac{a_{c}\left(\frac{p_{c}\tau}{p_{c}+1}\right)^{\frac{1}{a_{s}}}-a_{s}(2p_{c}+1)\left(\frac{1}{p_{c}(\text{\ensuremath{p_{c}}}+1)}\right)^{\frac{1}{a_{c}}}}{a_{c}a_{s}p_{c}(p_{c}+1)}
\]

Now setting $a_{c}=a_{s}=\frac{1}{2}$:

\[
\frac{-1+\tau}{(1+p_{c})^{2}}=\frac{2\left(p_{c}^{4}\tau^{2}-2p_{c}-1\right)}{p_{c}^{3}(\text{\ensuremath{p_{c}}}+1)^{3}}
\]

\[
\tau-1=\frac{2\left(p_{c}^{4}\tau^{2}-2p_{c}-1\right)}{p_{c}^{4}+p_{c}}
\]


\part*{Part 4. Analysis and Discussion}

4.1. For $\tau=1$, $p_{c}^{*}=1.39$, for $\tau=1/2$ , $p_{c}^{*}=1.59$,
the decrease of revenue due to the policies of Spotify, make the optimal
price of concerts go up. 

4.2. Taylor S's has monopoly power in concerts, that means she can
set prices of concerts, however she faces decreasing marginal revenue
in prices, optimality for Taylor S's profit maximization happen at
concert price where marginal revenue is equal to marginal cost. For
consumers an increase in price of concerts reduce their consumption
of concerts and songs due to substitution and income effects respectively.
But the gains in revenue of the price increase more than compensate
this losses for Taylor S's and she increases the prices nevertheless. 
\end{document}
